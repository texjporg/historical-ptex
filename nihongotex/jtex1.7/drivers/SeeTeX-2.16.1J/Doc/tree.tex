
 %  These macros are taken from an article called "Trees in TeX",
 %  written by David Eppstein, and published in TUGboat, Volume 6,
 % (1985), No. 1, pp. 31--35.
 
\newbox\treebox
\def\tree{\global\setbox\treebox=\boxtree}
\def\subtree{\ettext \boxtree}
\def\leaf#1{\subtree#1\endsubtree}
 
\def\endsubtree{\ettext \egroup}
\def\endtree{\endsubtree \settreesizes \typesettree}
 
\newif\iftreetext\treetextfalse         % Whether still aligning text
\def\boxtree{\hbox\bgroup               % Start outer box of tree or subtree
  \baselineskip 2.5ex                   % Narrow line spacing slightly
  \tabskip 0pt                          % No spurious glue in alignment
  \vbox\bgroup                          % Start inner text \vbox
  \treetexttrue                         % Remember for \ettext
  \let\par\crcr \obeylines              % New line breaks without explicit \cr
  \halign\bgroup##\hfil\cr}             % Start alignment with simple template
\def\ettext{\iftreetext                 % Are we still in inner text \vbox?
  \crcr\egroup \egroup \fi}             % Yes, end alignment and box
 
\def\cons#1#2{\edef#2{\xmark #1#2}}     % Add something to start of list.
\def\car#1{\expandafter\docar#1\docar}  % Take first element of list
\def\docar\xmark#1\xmark#2\docar{#1}    % ..by ignoring rest in expansion.
\def\cdr#1{\expandafter\docdr#1\docdr#1}% Similarly, drop first element.
\def\docdr\xmark#1\xmark#2\docdr#3{\def#3{\xmark #2}}
\def\xmark{\noexpand\xmark}             % List separator expands to self.
\def\nil{\xmark}                        % Empty list is just a separator.
 
\def\settreesizes{\setbox0=\copy\treebox  \global\let\treesizes\nil \setsizes}
\newdimen\treewidth                     % Width of this part of the tree.
\def\setsizes{\setbox0=\hbox\bgroup     % Get a horiz list as a workspace.
  \unhbox0\unskip                       % Take tree, unpack it into horiz list
  \inittreewidth                        % Get old width at this level.
  \sizesubtrees                         % Recurse through all subtrees.
  \sizelevel                            % Now set width from remaining \vbox.
  \egroup}                              % All done, finish our \hbox.
 
\def\inittreewidth{\ifx\treesizes\nil   % If this is the first at this level
  \treewidth=0pt                        % ..then we have no previous max width.
\else \treewidth=\car\treesizes         % 0therwise take old max level width
  \global\cdr\treesizes                 % ..and advance level width storage
  \fi}                                  % ..in preparation for next level.
 
\def\sizesubtrees{\loop                 % For each box in horiz list (subtree)
  \setbox0=\lastbox \unskip             % ..pull it off list and flush glue.
  \ifhbox0 \setsizes                    % If hbox, it's a subtree - recurse
  \repeat}                              % ..and loop; end loop on tree text.
 
\def\sizelevel{\ifdim\treewidth<\wd0    % If greater than previous maximum
  \treewidth=\wd0 \fi                   % Then set max to new high
\global\cons{\the\treewidth}\treesizes} % In either case, put back on list
 
\newdimen\treeheight                    % Height of this part of the tree.
\newif\ifleaf                           % Tree has no subtrees (is a leaf).
\newif\ifbotsub                         % Bottom subtree of parent.
\newif\iftopsub                         % Top subtree of parent.
\def\typesettree{\medskip \maketree \medskip}  % Make whole tree with spacing.
\def\maketree{\hbox{\treewidth=\car\treesizes  % Get width at this level.
  \cdr\treesizes                        % Set up width list for recursion.
  \makesubtreebox\unskip                % Set \treebox to text, make subtrees.
  \ifleaf \makeleaf                     % No subtrees, add glue.
  \else \makeparent \fi}}               % Have subtrees, stick them at right.
 
{\catcode`@=11                          % Be able to use \voidb@x.
\gdef\makesubtreebox{\unhbox\treebox    % Open up tree or subtree.
  \unskip\global\setbox\treebox\lastbox % Pick up very last box.
  \ifvbox\treebox                       % If we're already at the \vbox
    \global\leaftrue \let\next\relax    % ..then this is a leaf.
  \else \botsubtrue                     % Otherwise, we have subtrees.
    \setbox0\box\voidb@x                % Init stack of processed subs
    \botsubtrue \let\next\makesubtree   % ..and call \maketree on them.
  \fi \next}}                           % Finish up for whichever it was.
 
\def\makesubtree{\setbox1\maketree      % Call \maketree on this subtree.
  \unskip\global\setbox\treebox\lastbox % Pick up box before it.
  \treeheight=\ht1                      % Get height of subtree we made
  \advance\treeheight 2ex               % Add some room around the edges
  \ifhbox\treebox \topsubfalse          % If picked up box is a \vbox,
    \else \topsubtrue \fi               % ..this is the top, otherwise not.
  \addsubtreebox                        % Stack subtree with the rest.
  \iftopsub \global\leaffalse           % If top, remember not a leaf
    \let\next\relax \else               % ..(after recursion), set return.
    \botsubfalse \let\next\makesubtree  % Otherwise, we have more subtrees.
  \fi \next}                            % Do tail recursion or return.
 
\def\addsubtreebox{\setbox0=\vbox{\subtreebox\unvbox0}}
\def\subtreebox{\hbox\bgroup            % Start \hbox of tree and lines
  \vbox to \treeheight\bgroup           % Start \vbox for vertical rules.
   \ifbotsub \iftopsub \vfil            % If both bottom and top subtree
       \hrule width 0.4pt               % ..vertical rule is just a dot.
     \else \treehalfrule \fi \vfil      % Bottom gets half-height rule.
   \else \iftopsub \vfil \treehalfrule  % Top gets half-height the other way.
     \else \hrule width 0.4pt height \treeheight \fi\fi  % Middle, full height.
   \egroup                              % Finish vertical rule \vbox.
  \treectrbox{\hrule width 1em}\hskip 0.2em\treectrbox{\box1}\egroup}
 
\def\treectrbox#1{\vbox to \treeheight{\vfil #1\vfil}}
\def\treehalfrule{\dimen0=\treeheight   % Get total height.
  \divide\dimen0 2\advance\dimen0 0.2pt % Divide by two, add half horiz height.
  \hrule width 0.4pt height \dimen0}    % Make a vertical rule that high.
 
\def\makeleaf{\box\treebox}             % Add leaf box to horiz list.
\def\makeparent{\ifdim\ht\treebox>\ht0  % If text is higher than subtrees
  \treeheight=\ht\treebox               % ..use that height.
\else \treeheight=\ht0 \fi              % Otherwise use height of subtrees.
\advance\treewidth-\wd\treebox          % Take remainder of level width
\advance\treewidth 1em                  % ..after accounting for text and glue.
\treectrbox{\box\treebox}\hskip 0.2em   % Add text, space before connection
\treectrbox{\hrule width \treewidth}\treectrbox{\box0}} % Add \hrule, subs.
 
%%Examples
 
%\tree
% Tree
% Uses
% \subtree
%  Computer
%  Science
%  \subtree
%    Data
%    Structures
%    \leaf{Search tree}
%    \leaf{Priority queue}
%  \endsubtree
%\endsubtree
%\endtree
%
%\tree
%Start
%Game
%  \subtree
%  Enter
%   \leaf{(1,0)}
%   \leaf{(0,1)}
%  \endsubtree
%  \subtree
%  Don't enter
%    \subtree
%    Fight
%     \leaf{(0,1)}
%    \endsubtree
%    \subtree
%    Don't fight
%     \leaf{(1,0)}
%    \endsubtree
%  \endsubtree
%\endtree

