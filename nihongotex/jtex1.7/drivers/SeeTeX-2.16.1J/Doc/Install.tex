% Master File: Install.tex
% Document Type: LaTeX
\documentstyle[fullpage,12pt]{article}
\input{psfig.tex}
\input{tree.tex}
\def\tpic{{\sc tpic}}
\def\dviselect{{\sc dviselect}}
\def\xtex{{\sc xtex}}
\def\texx{{\sc texx}}
\def\texsun{{\sc texsun}}
\def\texb{{\sc tex3b1}}
\def\dvi{{\tt dvi}}
\def\iptex{{\sc iptex}}
\def\seetex{{\sc See\TeX}}
\def\mftobdf{{\sc mftobdf}}
\def\libtex{{\sc libtex}}

\def\makefile{{\tt Makefile}}
\def\fontdesc{{\tt fontdesc}}
\def\PK{{\tt PK}}
\def\GF{{\tt GF}}
\def\PXL{{\tt PXL}}
\def\TFM{{\tt TFM}}
\def\X{\sc X11}

\def\MetaFont{{\sc MetaFont}}

\def\imake{{\sc imake}}
\def\fontdesc{{\tt fontdesc}}

\def\Section#1#2{\section{#2}\label{section:#1}}
\def\SubSection#1#2{\subsection{#2}\label{section:#1}}
\def\SubSubSection#1#2{\subsubsection{#2}\label{section:#1}}


\def\BDF{{\tt BDF}}
\def\SNF{{\tt SNF}}
\def\PCF{{\tt PCF}}

% mctex global macros (shorthand, actually)
\def\mctex{MC-\TeX}
%\def\Unix{{\sc U\kern-1pt nix}}
\def\Unix{{\sc unix}}
\def\ps{{\sc PostScript}}
\def\dvi{{\tt DVI}}
\def\dviselect{{\tt dviselect}}
\def\gf{GF}
\def\pk{PK}
\def\pxl{PXL}
\def\tfm{TFM}
\def\cccount{{\tt\char`\\ count}}
\def\ccount#1{{\tt\char`\\ count#1}}

% `column style' options for @-specifiers,
% used as, e.g., \begin{tabular}{@{\coltt}ll}.
% For some reason these macros do not work for centered columns.
\def\colstyle#1{\hspace{\tabcolsep}#1}
\def\coltt{\colstyle\tt}
\def\colem{\colstyle\em}

\title{SeeTeX Installation Guide}
\author{
Dirk Grunwald \\
Dept. of Computer Science\\
University of Colorado\\
{\tt grunwald@foobar.colorado.edu}\\
\and
Chris Torek\\
Department of Computer Science\\
University of Maryland\\
College Park, MD 20742\\
{\tt chris@mimsy.umd.edu}\\
\thanks{To be fair, Chris didn't write this slop, except for the sections
on the {\tt fontdesc} file; I include his name here because of that section
and because he wrote the 'mctex' library on which SeeTeX stuff is based,
not to mention some of the tools herein}
}
%%
\date{1990}
\begin{document}
\maketitle

%\begin{abstract}
%This documents the installation of the {\seetex} package, including
%{\xtex}, {\dviselect}, {\texx}, {\texsun}, {\texb},
%{\iptex} and {\mftobdf}.
%\end{abstract}
%
{\footnotesize \tableofcontents} 
%\newpage

\section{Introduction}

{\seetex} is a package of output drivers for {\TeX}, and includes the
following programs:

\begin{center}
\begin{tabular}{|l|p{5in}|}
\hline
Program & Purpose \\
\hline
{\xtex} & 
	 Fast {\X} windows based {\TeX} previewer
	(formerly called {\texx2}). \\
{\mftobdf}&
	 Program to convert {\MetaFont} generated fonts to
	the {\BDF} format; this is used by {\xtex}. \\
{\texx}&
	 Antiquated raw {\X} windows based {\TeX} previewer. \\
{\texsun}&
	 SunView based {\TeX} previewer. \\
{\texb}&
	 A {\TeX} previewer for the AT\&T 3B1,
	contributed by Andy Fyfe at CalTech.
\\
{\dviselect}&
	 Utility to extract individual pages from a {\dvi} file.
	This is just Torek's {\dviselect}. \\
{\iptex}&
	Output driver for Imagen printers; this is a very different 
	and enhanced version of Torek's {\iptex}. \\
\hline
\end{tabular}
\end{center}

All of these programs depend on a library of routines, called {\libtex},
that was originally written by Chris Torek at the Univ. of Maryland.
{\seetex} is distributed in two ways.
The first, usually called
something like {\verb|SeeTeX-2.15.3.tar.Z|} is the full {\seetex} distribution.
The second, usually called \verb|xtex-2.15.3.tar.Z| is
the partial {\xtex} distribution that updates the version of {\xtex} in the
full {\seetex} distribution. Because the full {\seetex} distribution
changes infrequently, snagging only the {\xtex} distribution allows
you to update {\xtex} without having to grab the full {\seetex} tar.
In this installation guide, I will assume you have unpacked full 
{\seetex} distribution and have the directory structure shown in
Figure~\ref{figure:directory} in the following document.  Only
directories and important files are included in
Figure~\ref{figure:directory}.


\begin{figure}[t]

\centerline{%
\small
\tt
\tree{SeeTeX}
  \leaf{Imakefile}
  \leaf{Imake.Config}
  \leaf{Makefile.raw}
  \leaf{fontdesc-example}
  \subtree{Doc}
    \endsubtree
  \subtree{Dviselect}
     \leaf{Imakefile}
     \leaf{Makefile.raw}
    \endsubtree
  \subtree{Iptex}
     \leaf{Imakefile}
    \endsubtree
  \subtree{Texsun}
     \leaf{Imakefile}
     \leaf{Makefile.raw}
    \endsubtree
  \subtree{Texx}
     \leaf{Imakefile}
    \endsubtree
  \subtree{Xtex}
     \leaf{Imakefile}
     \leaf{Xtex.ad}
    \endsubtree
  \subtree{libtex}
     \leaf{Imakefile}
     \leaf{Makefile.raw}
    \endsubtree
\endtree
}

\caption{{\seetex} Directory Structure}  
\label{figure:directory}
\end{figure}

\SubSection{intro:steps}{How Not To Read This Manual}

{\seetex} is configured to use the {\imake} installation configuration
system distributed with {\X}. If you do not use {\imake}, you can
only install {\libtex}, {\texsun} and {\dviselect} using the
handcrafted makefiles labeled {\tt Makefile.raw}. A portion of the
following comments related to configuration and installation will be
useful; the majority will not. If you run {\X} and do not use
{\imake}, contact your vendor or system supportoid to get {\imake}
installed.

If you already understand {\TeX} fonts and the {\fontdesc} file,
you can skip \S\ref{section:tex fonts} and \S\ref{section:fontdesc},
resp.
If you are installing {\texsun} and {\dviselect} using the raw
makefiles (i.e., not using {\imake}), skip to
\S\ref{section:raw makefiles}.
If you are using {\imake}, and are installed everything
but {\xtex}, you only need to read \S\ref{section:all but xtex}.
If you're also installing {\xtex} and {\mftobdf}, you need to
read \S\ref{section:xtex}.


\Section{tex fonts}{A Little Explanation of {\TeX} Fonts}

When printing {\TeX} documents, you use {\em bitmap} fonts that can be
stored in the {\PK}, {\GF} and {\PXL} formats.
The bitmaps are {\em designed} for a specific size, e.g., {\tt cmr10} is
a 10 point Computer Modern Roman font. This font can be {\em rendered}
at different {\em resolutions} and {\em magnifications}.
For example, on a laser printer, you might render the {\tt cmr10} font
at 300dpi (dots per inch).
For a computer display, you might render {\tt cmr10} at e.g., 85dpi
rather than 300dpi.
Magnifications are expressed as the percentage magnification multiplied
by 1000 and then truncated,
so the normal magnification is $1000$ (i.e. magnify by 1.0).

For the {\GF} and {\PK} formats, a {\em file suffix} is attached to
each font file encoding the resolution and magnification. A similar,
although completely different suffix is appended to the antiquated
{\PXL} format.  If you still use {\PXL}, flog your local system
supportoid and convert those fonts to {\GF} or {\PK} -- you'll save
oodles of disk space.
But lets say you render {\tt cmr10} for a 300dpi print at
magnifications of 500, 1000, 1098, 1200, 1440, 1728, 2074 and 2489.
These numbers are ``magic'' for {\TeX} and {\LaTeX}, because they
magnify the 10pt font by $1220$ at each step.
This renders our nominal 10pt font to 5pts, 10pts, 11pts, 12pts, 14pts,
17pt, 20pt and 24pt. See the {\TeX}book for more details.
In general, a 10pt font rendered to look like a 5pt font doesn't look
as good as a font designed to be a 5pt font.

For each of these magnifications, we'll get a file with a suffix
that's approximately
$$
	\mbox{suffix} = \mbox{resolution} \times
		\frac{\mbox{magnification}}{1000}
$$
and ends in either {\PK} or {\GF}. So, for our buddy {\tt cmr10}
rendered at 300dpi and stored in {\PK} format, we'd have the following files:
\verb|cmr10.150|,
\verb|cmr10.300|,
\verb|cmr10.327|,
\verb|cmr10.432|,
\verb|cmr10.518|,
\verb|cmr10.622|,
\verb|cmr10.747|.
%
Whether you have all these files or not is up to you, your {\TeX}
administrator and your disk.  Only programs that print {\TeX} documents
need these bitmap fonts; {\TeX} itself uses something called {\TFM}
files that record the width and height of each
glyph in a font. This is enough information for {\TeX} to layout your
document. There is only {\em one} {\TFM} file for each font, e.g.  for
{\tt cmr10}. You do not need different {\TFM} files for each
magnification or resolution of that font, because the {\TFM} distances
are encoded in units of $10^{-7}$ meters. Magnifications of the font
simply scale those dimensions.  Different resolutions affect only the
bitmap fonts, and not the absolute dimensions recorded in the {\TFM}
file; the 'x' in a 10pt font is always 10pt wide, or $10 \times
\frac{254}{7227}$ cm wide, no matter what the actual {\em resolution}
of the printing device. You are now a {\TeX} font wizard.

\Section{fontdesc}{Configuring and Installing \protect{\fontdesc}}

When you edit \verb|SeeTeX/Imakefile|, you will be asked to
select a destination for the {\fontdesc} file. Pick one now,
and copy \verb|SeeTeX/fontdesc-example| to
that destination and edit the file.

{\bf Note: \em The remainder of this section is taken from the
\mctex\ library documentation on the fontdesc file.}

The font description file provides a sequence of directories to
search for fonts.
A short fontdesc is shown in Table~\ref{table:fontdesc}
(although you should probably start with the longer one
in {\tt misc/fontdesc}).
Normally, all fonts are stored in a directory hierarchy.
The top-level directory contains the {\tt .tfm} files for all fonts
and a subdirectory for each font
containing all the magsteps for that font.
The fontdesc file also determines which font storage formats are used.
The library is capable of reading all three standard \TeX\ font formats
(\pk, \gf, and \pxl).
The \pk\ format is the best of the three.
\gf\ files are the slowest to read,
and take an intermediate amount of disk space.
\pk\ files are the smallest,
and intermediate in speed.
\pxl\ files are the fastest to read,
but consume about six times as much storage
as \pk\ fonts.
For a reasonable set of fonts,
this makes the difference between 10 megabytes
and 60 megabytes:
the space difference is great, and
the speed difference is only marginal.
In addition,
\pxl\ formats are now considered obsolete
and support for them may be dropped in the future
without warning.
%
\begin{table*}
\small
\setbox0=\vbox{\tt\catcode`\%=12 \catcode`\#=12 \obeyspaces
\hbox{#       TYPE    SPEC    SLOP    PATH}
\hbox{font    pk      *       2       %f.%mpk}
\hbox{}
\hbox{fontenv pk      *       2       TEXFONTS        %f.%mpk}
\hbox{}
\hbox{font    pk      *       2       /usr/local/lib/tex/fonts/%f/%f.%mpk}
\hbox{font    gf      *       2       /usr/local/lib/tex/fonts/%f/%f.%mgf}
\hbox{}
\hbox{font    box     *       0       %f.tfm}
\hbox{font    box     *       0       /usr/local/lib/tex/fonts/%f.tfm}
}
%\tracingonline1
%\showboxbreadth=999 \showboxdepth=1
%\showbox0
\centerline{\box0}
\caption{A Sample {\tt fontdesc}}
\label{table:fontdesc}
\end{table*}
%

In addition to the three standard font formats,
the library supports several `special' font formats.
They all work by reading a \tfm\ (\TeX\ font metrics) file.
The {\tt tfm} font format
(also known as {\tt invis})
carries only width information.
It is intended to be used for printer-resident fonts,
such as those that can be used by the \ps\ driver.
Since it has no rasters,
it prints out as blank space,
and so it can also be used
in its {\tt invis} guise
in place of S{\small LI}\TeX\ `invisible' fonts,
to save disk space.
Printouts that use the invisible font format
may have placement errors
(of up to {\em maxdrift\/} pixels;
see \S\ref{sec:drift}, p.~\pageref{sec:drift}),
so using a real font is preferable.
The {\tt blank} font format also prints out as blank space,
but carries with it a warning that the desired font magnification
was not found.
The {\tt box} format prints out as little rectangular boxes,
rather like those in Chapter~11 of {\em The \TeX book},
showing where \TeX\ thinks each character went.
Like blank fonts, it carries a warning.
The blank and box formats
are intended as `last resorts';
they exist
only so that a \dvi\ file that uses fonts in unavailable magnifications
can still be printed (with the offending glyphs missing).
If neither box nor blank fonts appear in the font description file,
such files cannot be printed at all.

\tfm\ fonts will always match at any size.
They must therefore appear late in the fontdesc file,
to keep them from `taking over' other fonts
whenever that is not desired.
Of course, when it {\em is\/} desired,
such as for built-in \ps\ fonts,
it should appear before other fonts so that it will indeed `take over'.
If nothing else,
this will improve performance
(since the conversion programs
will not have to hunt for nonexistent font files).

There are two kinds of font definition lines,
beginning with the words {\tt font} and {\tt fontenv}.
(Lines that start with any other word are ignored;
all other words are reserved for future expansion.)
The {\tt fontenv} form differs only slightly from that of {\tt font},
and its use will be described in a moment.
For a {\tt font},
there should be four more words following the word {\tt font}.
They are interpreted as follows:
\begin{description}
\item[type]
This is one of the seven font types
({\tt blank}, {\tt box}, {\tt gf},
{\tt invis}, {\tt pk}, {\tt pxl}, {\tt tfm}).

\item[spec]
This is a {\em print engine specifier}.
\TeX\ fonts come in several varieties.
The major kinds are `write black' and `write white'.
Each \dvi\ driver has a print engine specifier
(some have several);
only if the program's specifier matches that in the font description
will that font be used.
Thus, if you have both a Canon-based (write-black) printer
and a Ricoh-based (write-white) printer,
you can put the write-black and write-white fonts
in different subdirectories
and make sure that each driver
gets only the kind of fonts it uses.
An asterisk `{\tt *}' in the font description file
will match any print engine specifier;
this will suffice for many installations.

\item[slop]
The slop factor is used to make up for
the sins of finite-precision arithmetic.
Fonts come in various magnifications,
normally denoted as {\em magsteps}.
Each magstep corresponds to a power of 1.2.
For instance, \verb|\magstep4| is $1.2^4$, or $2.0736$.
These numbers are rounded to the nearest $.001$,
so that this becomes $2.074$.
Font files are named according to this magnification
multiplied by the number of dots per inch for that font.\footnote
{For \pxl\ files, the number is then multiplied by 5,
for reason which are neither good nor worth explaining.}
For instance, the \gf\ file for Computer Modern Roman, ten point, at
\verb|\magstep4| on a 300 dpi printer,
when both the \dvi\ file and user magnifications are 1000,
is called \verb|cmr10.622gf|.
The $622$ comes from the formula
$300 \cdot 2.074 \cdot 1000/1000 \cdot 1000/1000 = 622.2$.
In the course of all this arithmetic,
the computer has a tendency to get the last digit wrong.
The slop factor is used to correct for this.
The font library will look for magnifications
of up to $\pm\hbox{\rm slop}$ away from the number it computed.
Thus, a slop of 1 will allow an off-by-one error to slip by.
A slop of 0 requires an exact match.
Since fake fonts---blank, box, and invis---have no inherent magnification,
they should be given no slop;
others should have as little slop as possible
(the more there is, the slower things get).

\item[path]
The library builds the path based on a {\em format\/} specifier.
Characters other than the percent sign `{\tt \%}' are copied literally.
The percent sign introduces a {\em conversion},
similar to those in the C library `printf' routines.
A single percent sign can be obtained by doubling it, {\tt \%\%}.
The other conversions are as follows (the examples in
parentheses are for a font like \verb|cmr10 scaled \magstep4|
on a 300 dpi printer):
{\tt \%f}, which gets the name of the font ({\tt cmr10});
{\tt \%m}, which gets the magnification ({\tt 622});
and {\tt \%b}, which gets the {\em base name\/} of the font,
like {\tt \%f}, but omitting any numeric suffix ({\tt cmr}).
These three suffice to express most font organisations.
More formats may be added, if necessary, by changing {\tt lib/font.c}
(see the routine {\tt pave()}\footnote
{So named because it takes an abstract path (with percent escapes)
and makes it concrete, `paving it over' (as it were).
Well, I have been told never to explain a pun\ldots.}).
\end{description}

A {\tt fontenv} is almost exactly the same as a {\tt font},
except that the fifth word is the name of an environment variable
rather than an actual path format,
and that a sixth word (the {\em name part\/}) is required.
If the named variable is present in the environment
when the fontdesc file is being read,
it is split up like any other \Unix\ path
(i.e., at colons)
and each sub-pathname has the name part appended.
Each result is used as if it were the path portion
of a regular {\tt font}.
For instance, given the example fontdesc above,
if {\tt \$TEXFONTS} were set to {\tt /here:/there},
the line
\begin{quote}
\begin{verbatim}
fontenv pk      *       2       TEXFONTS        %f.%mpk
\end{verbatim}
\end{quote}
would make \mctex\ programs act as if the pair of lines
\begin{quote}
\begin{verbatim}
font    pk      *       2       /here/%f.%mpk
font    pk      *       2       /there/%f.%mpk
\end{verbatim}
\end{quote}
were included instead.
Typically,
the name part of each {\tt fontenv} will simply be {\tt\%f.\%m\it type}.
All \tfm-based fonts, however,
must have {\tt\%f.tfm} appended instead,
so that, e.g., a {\tt fontenv invis} for {\tt iamsss8} looks for 
{\tt iamsss8.tfm} rather than {\tt iamsss8.0invis}.

The startup time of \mctex\ drivers
is heavily influenced by the amount of directory searching done.
A long fontdesc file,
or a fontdesc file that uses long environment paths,
can slow these programs severely.
A {\tt fontenv} whose variable is not present in the environment
is ignored,
at essentially zero cost in speed,
so one way to reduce startup time
is to use a different environment variable for each kind of font.
For instance,
by looking for \pk\ fonts in {\tt\$PKFONTS}
and for \gf\ fonts in {\tt\$GFFONTS},
users with only \gf\ fonts can avoid one of the searches entirely.
In all cases,
the total amount of searching
can also be reduced by using a minimal `slop' value.

Searching stops as soon as a usable font is found,
so putting likely candidates first is also a good idea.
User directories tend to be less likely places to find font files,
so moving the main system directory to the first line will help.
Unfortunately,
this also makes it hard
for users to override the standard font files---they
will have to use different names for the fonts,
or set {\tt\$TEXFONTDESC} to name a different fontdesc file.\footnote
{The natural consequence of the latter is that,
some years later,
the fonts will be moved to some other directory,
and the system fontdesc updated,
and then several irate users will complain that the software is broken,
having forgotten that they copied and altered
the---now out of date---fontdesc file.
The administrator will demonstrate that the software works;
but the users will insist it does not.
After several hours of probing about,
someone will finally recall this footnote,
and the mystery will at last be explained.
In another few years, the cycle will then repeat.}
A reasonable compromise can be had
by using a {\tt fontenv} line
to refer to an unlikely environment variable name.
Those users who {\it want\/} to override the standard font files
can then set this variable,
and everyone else will enjoy faster font lookups.

\Section{all but xtex}{Configuring \&
	Installing Basic {\seetex} Programs Using {\imake}}

To install the basic programs and libraries, you need to edit
\verb|SeeTeX/Imakefile| and install the font description
file, {\fontdesc}. \S\ref{section:fontdesc} describes the latter.

Edit the file \verb|SeeTeX/Imake.Config|. This file contains
configuration parameters you need to tweak.
Look for the following lines:

\begin{verbatim}
FontDesc=/usr/local/lib/tex82/fontdesc
IprBin=/usr/local/bin
\end{verbatim}

The \verb|FontDesc| variable names the location of a file, called
{\fontdesc}, that describes the location of {\TeX} fonts
on your machine. This file is usually in your
standard {\TeX} library location; edit \verb|FontDesc| to point to your
local library. If you are only going to use {\xtex} (and not {\mftobdf})
you do not need to worry about \verb|FontDesc|.

The \verb|IprBin| program indicates where the \verb|ipr| program lives;
this is an interface to Imagen printers. If you will not be using {\iptex},
ignore this line; otherwise, change \verb|IprBin| to point to the \verb|ipr|
binary.

Look in \verb|SeeTeX/Imakefile| for the line: 
\begin{verbatim}
SUBDIRS = libtex Xtex Mftobdf Dviselect Texx Iptex Fonts Texsun  
\end{verbatim}
Remove the names of any programs from the \verb|SUBDIRS| line that you
do not wish to build. You must include \verb|libtex|, because it is
used by all programs. {\texsun} still depends on SunView, and is not
(yet) compatible with the more recent OpenLook X-windows interface.

Once you have changed \verb|Imakefile|, execute the following commands:

\begin{verbatim}
[SeeTeX/] xmkmf
[SeeTeX/] make Makefile
[SeeTeX/] make Makefiles
\end{verbatim}

Throughout this document,
the command prompt, e.g., \verb|[SeeTeX/]|,
indicates the directory within which the
command should be executed.  The variable \verb|TOP| should point to
your {\X} library directory; this is usually installed in
\verb|/usr/local/{\X}|, but may be installed in different places. If
get error messages from \verb|imake|, ask your local {\X} guru where the
configuration directory lives.

If you're not installing {\xtex}, you should be able to run
\begin{verbatim}
[SeeTeX/] make
\end{verbatim}
at this point, or alternatively,
\begin{verbatim}
[SeeTeX/] make install
[SeeTeX/] make install.man
\end{verbatim}
if you're confident that everything is correctly configured.
This will build the library ({\libtex}) and each selected program.

\Section{x fonts}{A Little Explanation of {\X} Fonts}

{\xtex} opens another can of worms, because it uses
{\X} fonts that are created from {\TeX} fonts via {\mftobdf}.
First, we must digress to talk about {\X} fonts.
{\X} is a {\em network window system}; {\em clients} send messages to
{\em servers} across a network, causing the server to draw things on
the display.
{\xtex} is a {\em client}.
The {\em server} runs on your workstation, PC or terminal.

\begin{figure}
\centerline{\psfig{figure=Xfonts.ps}}
\caption{Schematic Usage of Fonts in {\X} windows}  
\label{figure:schematic}
\end{figure}


{\xtex} displays {\TeX} output by telling the server to display
specific fonts within a window. Those fonts correspond to {\TeX}
fonts, but they, like all {\X} fonts, must live on the server.
Figure~\ref{figure:schematic} shows a schematic representation
of the information needed and provided by each program.
{\xtex} uses information from the {\TFM} file as well as the {\X} font
to adjust the placement of individual letters on the display to
match the proper {\TeX} output within a single pixel of your display.
If your server does not have the fonts that {\xtex} asks for, it will
use a {\em default font} in their stead. This will look messy, but
you'll still be able to preview your document.
{\mftobdf} converts existing {\em bitmap} fonts to {\BDF} files,
which can then be installed on your {\X} window system ``font system''.
For workstations, this is usually your normal filesystem or a shared
filesystem.

\SubSection{installing x fonts}{Installing {\X} Fonts}

Typically, fonts are pre-installed in one of several directories
that are listed in the server {\em font path}.
Each directory in the font path normally contains a file,
usually called {\tt fonts.dir},%
%%
\footnote{All of this ``usually'' and ``normal'' stuff is there for a reason.
For example, Ultrix UWS2.0 on a VAXstation I previously used didn't use a
{\tt fonts.dir} file, and scanned all the files in each directory every time
it started X. Very slow, to be sure}%
%%
that maps between {\em font names} and {\em file names}.
For example, the font named {\tt cmr10.85} might live
in the any of the files {\tt cmr10.85.bdf}, {\tt cmr10.85.snf}
or {\tt cmr10.85.snf.Z}. The {\tt fonts.dir} file tells the
{\X} server where to find the font with the given name.

{\X} fonts are distributed in {\BDF} files, or {\em Bitmap Distribution
Format} files; however, individual servers convert these portable
{\BDF} files to {\em server normal format},
or {\SNF} files.
Most servers can read:
\begin{enumerate}
\item raw {\BDF} files (e.g., {\tt cmr10.85.bdf})
\item compressed {\BDF} files (e.g., {\tt cmr10.85.bdf.Z})
\item raw {\SNF} files (e.g., {\tt cmr10.85.snf})
\item compressed {\SNF} files (e.g., {\tt cmr10.85.snf.Z})
\end{enumerate}
The {\SNF} files are {\em architecture specific}; in other words, you
may not be able to use the same set of {\SNF} files on all of your
systems. I've found that {\SNF} files can be shared between
Sun-3 and Sun-4, but not e.g., the Sun-i386.

Certain {\X} servers, e.g., DEC {\X} servers,
use the {\PCF} format rather than {\SNF}; these
servers can read files like
{\tt cmr10.85.pcf} or
{\tt cmr10.85.pcf.Z}.
Unlike {\SNF} files, {\PCF} files are {\em architecture independent};
a single set of {\PCF} files suffice for all servers that can read
the {\PCF} format. Sadly, this appears to be limited to the DEC {\X}
servers.%
%%
\footnote{I had made a brief stab at having the X server load {\PK} file
directly. It's mostly complete,
and if you're interesting in finishing that work, send me mail.}%
%%

Let's assume you've been given a {\BDF} file describing
a {\TeX} font, and that  file is called {\tt cmr10.85.bdf}.
To install this new font for {\X}, follow these steps:

\begin{enumerate}
  
\item First, convert the {\BDF} file to either a {\SNF} file
	(stock MIT {\X} windows) or a {\PCF} file (DEC).
%
\begin{description}
\item[{\SNF} file]: use the {\tt bdftosnf} program that should come with
	your {\X} windows system. For example, if you have the file
	{\tt cmr10.85.bdf}
	execute
\begin{verbatim}
    bdftosnf cmr10.85.bdf > cmr10.85.snf
\end{verbatim}

\item[{\PCF} file]: use the {\tt dxfc} program that should come with
	your {\X} windows system. For example, if you have the file
	{\tt cmr10.85.bdf}
	execute
\begin{verbatim}
    dxfc cmr10.85.bdf > cmr10.85.pcf
\end{verbatim}

\end{description}

\item Second, decide if you want to compress the {\SNF}/{\PCF} file.
For fast workstations, like SPARC or DEC/RISC systems, you should probably
compress the output file using {\tt compress}; you save a lot of disk space
and won't notice the time spent uncompressing the file.

\item	Now, pick a directory in which you want to install the font
	and copy the font to that directory.
	You can either use a directory already on the {\X} font path,
	or you can make a new directory and add that your font path.

	If you make a new directory, you must add it to your font path using
	the {\tt xset} program, e.g. if you added the directory
	/usr/lib/X11/fonts/tex, you should say:
\begin{verbatim}
[unix] xset +fp /usr/lib/X11/fonts/tex
\end{verbatim}
	{\em every time you start your {\X} server}. You can optionally
	change your {\X} startup scripts to always include this font path.

	\begin{quotation}\noindent
	 {\tt N.B.} The stock DEC {\X} does not include include {\tt xset},
	but you can put the {\PCF} files in either
\verb|/usr/lib/X11/fonts/local/75dpi| if you use a 19 inch monitor
or 
\verb|/usr/lib/X11/fonts/local/100dpi| if you use a smaller monitor.
These directories are in your default font path.
	\end{quotation}

\item	You must rebuild the {\tt fonts.dir} file in the directory
	where you placed that font. Go to the directory and execute
	either {\tt mkfontdir} (stock MIT {\X} windows) or
	{\tt dxmkfontdir} (DEC {\X} windows).

	This will take a while, because each file in the directory
	is examined to see if it's a font, and the font name is
	added to the {\tt fonts.dir} file.

\item	Now, tell your server that the font directories have changed,
	and that it should rescan the {\tt fonts.dir} files. Do this
	using \verb|xset fp rehash|.

\item	Lastly, check that the font was properly installed. The
	{\tt xlsfonts} program lists all the fonts your server knows
	about. So, for our sample font, called {\tt cmr10.85},
	execute \verb/xlsfonts | grep cmr10.85/ and see if it's
	listed. You can also execute \verb|xfd -fn cmr10.85| to
	see the font.
\end{enumerate}

If you have a bevy of {\BDF} files to convert, you may want to use the
shell scripts {\tt BDFZtoSNFZ} or {\tt BDFZtoPCFZ} to quickly convert
compressed {\BDF} files to either compressed {\SNF} or {\PCF} files.
These shell scripts are distributed with {\xtex} in the
\verb|SeeTeX/Fonts| directory.


Currently, host \verb|foobar.colorado.edu| provides a large set of
{\BDF} files for 85dpi fonts, rendered at various magnifications.
You can {\tt FTP} those files or generate your own {\BDF} files using
{\mftobdf}; see \S\ref{section:mftobdf}.

\Section{xtex}{Installing {\xtex}}

Given all of this, the installation of {\xtex} is fairly simple.

The first thing you must do is decide what sizes of fonts you're going
to be using. As delivered, {\xtex} defaults to 85dpi fonts. The two
options are either screen resolution
fonts (e.g., the 85dpi fonts) generated by {\MetaFont}
specificly for {\xtex}, or 100dpi fonts ``shrunken'' from existing
printer fonts, usually 300dpi or 400dpi.

The advantage of the screen resolution fonts is that they're easier
to read. The disadvantage is that you may need to use {\MetaFont} to
build the fonts at those sizes.

The advantage of shrunken printer fonts is that it's easy to build
them, because you probably already have them. The disadvantage is that
they look kind of crappy and are harder to read.

\begin{enumerate}
\item
	Edit file \verb|SeeTeX/Xtex/Xtex.ad|, the application defaults file,
	and check the line that looks like:
\begin{verbatim}
*.tfmPath: .:/usr/local/lib/tex82/fonts:/usr/local/lib/tex/fonts:
/usr/lib/tex82/fonts:/usr/lib/tex/fonts:/usr/local/lib/tex/tfm:
/usr/local/Tex/lib/tex/tfm
\end{verbatim}
	This describes the path that {\xtex} will use to locate {\TFM} files.
	If your default {\TFM} location is not in that path, add it.
	An alternative is to set the {\tt TEXFONTS} environment variable
	to a list similar to this. This environment variable is also used by
	{\TeX}.

\item	If you want to use something other than the default 85dpi fonts,
	edit \verb|SeeTeX/Xtex/Xtex.ad| and comment in the following lines,
	and remove the lines that follow them.
\begin{verbatim}
#	*.mag:333
#	*.smallMag:333
#	*.largeMag:500
#	*.dpiHoriz:300
#	*.dpiVert:300
\end{verbatim}

\item	You may also want to look through the rest of
	\verb|SeeTeX/Xtex/Xtex.ad| and edit things to your liking.
	Users can override the defaults using their own resource files.

\item	Edit \verb|SeeTeX/Imake.Config|.
\begin{enumerate}

\item If your system provides Adobe Display Postscript,
	include the line 
\begin{verbatim}
	#define DISPLAY_POSTSCRIPT
\end{verbatim}
	otherwise, include the line
\begin{verbatim}
	#undef DISPLAY_POSTSCRIPT
\end{verbatim}
	If you have Display Postscript, but are not using the DEC
	version thereof, you may need to fiddle with the Imakefile.

\item	{\xtex} uses several {\em popup windows} to ask you things;
	by default, \goodbreak
	these are {\tt overrideShellWidetClass} windows. \goodbreak
	Some people prefer {\tt transientShellWidgetClass} windows,
	because they can be moved
	around by the user. Select whichever one you prefer.
\end{enumerate}
\end{enumerate}

Now, return to the top level of the directory tree and execute:

\begin{verbatim}
[SeeTeX/] make Makefiles
[SeeTeX/] make install
\end{verbatim}

Note: you cannot ``test drive'' {\xtex}. It depends very heavily on the
application defaults file, {\tt Xtex.ad} being installed in the proper place,
typically {\tt /usr/lib/X11/app-defaults/Xtex}. If this file is not installed,
or if an older version is installed, you will get diagnostic messages
and xtex will refuse to run.

\Section{mftobdf}{Installing and Using {\mftobdf}}

Installing {\mftobdf} is relatively stock, and it probably built just
fine if you followed the steps in \S\ref{section:all but xtex}.
The hard part is using it to build {\BDF} files.
First, read \S\ref{section:tex fonts} and \S\ref{section:x fonts}
and \S\ref{section:installing x fonts}.
Second, determine if you're going to use screen-resolution fonts
or shrunken fonts (see \S\ref{section:xtex}).

\SubSection{screen fonts}{Using Screen Resolution Fonts}

There are two ways to get screen resolution fonts,
the easy way, the really hard way and the hard way.
Let's assume that you're going to generate fonts at XYZdpi, where
XYZ might be 118, 85 or whatever.

\SubSubSection{easy way}{The Easy Way}

Decide that you want to use 85dpi fonts, and FTP to 
\verb|foobar.colorado.edu|. Hunt around until you find a directory
\verb|Bdf-85dpi|. Go there, and grab all the fonts. Install them
(see \S\ref{section:installing x fonts}), and use them for a few
weeks. Now, go to your font directory, and do
\begin{verbatim}
[unix] ls -lu | head -100  > KEEP-THESE  
\end{verbatim}
and then remove all the fonts not listed in the \verb|KEEP_THESE| file.
Then, rebuild the \verb|fonts.dirs| file.
This gives you the top 100 fonts commonly used at your site.
If you are missing some fonts, just FTP to foobar again, or read the
next item.

\SubSubSection{hard way}{The Hard Way}

Lets assume that you have a slew of {\GF} or {\PK} fonts.
Perhaps you picked up 118dpi fonts from a {\TeX} library somewhere.
You need to convert those fonts to {\BDF} files.
The easiest way to do this is to go to the directory holding those fonts
and say:
\begin{verbatim}
[unix] mftobdf -dpi XYZ *gf *pk  
\end{verbatim}
and sit back. If you specify hundreds of fonts, {\mftobdf} may barf,
because there's a memory leak that I can't find and don't care about.
You can also specify, e.g,
\begin{verbatim}
[unix] mftobdf -dpi XYZ -mag 1098 cmr10
\end{verbatim}
to convert a single font. At 85dpi, this would be equivilent to saying
\begin{verbatim}
[unix] mftobdf -dpi 85 cmr10.93pk
\end{verbatim}

Note that you must {\em always} specify the dots per inch you're
using.  After you've generated the {\BDF} files, you can discard the
{\PK} or {\GF} files, if you like. {\xtex} does not use those directly.

\SubSubSection{really hard way}{The Really Hard Way}

What if you can't find a set of fonts
at your desired resolution on the network?
You need to build the font fonts using {\MetaFont},
and then convert them to {\BDF} format.

\begin{enumerate}
\item Find a {\MetaFont} wizard. All you need to know is there your
	configuration files are, and if they can show you how to build
	a font, all the better.

\item Make certain that you have a \verb|modedef| that looks something like:
\begin{verbatim}
mode_def sun =                  % bitgraph and sun screens (true size)
 proofing:=0;                   % no, we're not making proofs
 fontmaking:=1;                 % yes, we are making a font
 tracingtitles:=0;              % no, don't show titles in the log
 pixels_per_inch:=85;          % try it to see if it will do
 blacker:=.35;                  % make pens a bit blacker (Karl Berry's value)
 fillin:=.1;                    % and compensate for diagonal fillin
 o_correction:=.3;              % but don't overshoot much
 enddef;
\end{verbatim}
%
This is the \verb|modedef| I used to generate my 85dpi fonts.
Change the \verb|pixels_per_inch| parameter to XYZ. You may want to fiddle
with the other parameters as well. It's all shots in the dark.

\item Build a version of {\MetaFont} that understands this modedef, 
	i.e. use {\tt inimf} to build a new {\tt plain.base} file.

\item Build a sample font using, e.g.

\begin{verbatim}
[unix] mf '\mode:=sun; mag:= 1098/1000; \batchmode; \input cmr10'
\end{verbatim}
%
This should give you a file, \verb|cmr10.ABCgf|, where the suffix
should be $1098 * \frac{XYZ}{1000}$. If it's not, it's probably because
the \verb|modedef| isn't being recognized; check that you're using the
right \verb|plain.base| file. 

\item
Obviously, the hard part is generating all the needed {\GF} files.
The directory \verb|SeeTeX/Fonts| contains some utilities
that makes all this easier. See the file \verb|SeeTeX/Fonts/README|.

One thing to remember is that you probably want fonts at a {\em normal}
size and a {\em large} size. E.g., if you're looking at your document
at 85dpi and normal magnification of 1000, you might want to ``blow things up''
to e.g., 1440 magnification (the default in {\xtex})
to see details a little better. Thus, not only do you
want to generate e.g., {\tt cmr10} at 1000, you also want to generate it
at 1440.
Happily, the numbers that {\TeX} and {\LaTeX} use (1000, 1220, 1440, etc)
scale by $1.2$, and you typically don't need to generate a another complete
set at the larger magnification, because you'll already have most of them.
\end{enumerate}

Now, following the directions in \S\ref{section:hard way} to convert
these to {\BDF} files.

\SubSection{shrunken fonts}{Using Shrunken Fonts}

Converting your exsiting fonts is relatively easy. Go to the directory
containing your {\PK} or {\GF} fonts.
Font shrinking, or scaling, is specified similar to
magnifications; however, the only valid scalings are $1000/N$ where
$N$ is an integer.  Thus, $500$ and $333$ are most common. For 300dpi
fonts, these would give you roughly 150dpi and 100dpi versions of your
printer fonts. In {\xtex}, $500$ is normally used for the ``large''
magnification and $333$ is used for small magnification.

So, you generate the fonts at scaling XYZ from fonts that were
originally ABCdpi, say:
%%
\begin{verbatim}
[unix] mftobdf -dpi ABC -mag 1000 -scaled XYZ *pk *gf  
\end{verbatim}
%%

The {\tt -scaled} keyword is the important thing here. After you've
gotten a passle of {\BDF} files from all of this, convert them to your
local server format and install them according to the directions in
\S\ref{section:installing x fonts}.

\Section{raw makefiles}{If you do not use {\imake}}

If you don't use {\imake}, the only programs you can easily install
are those that have a file titled {\tt Makefile.raw}.
Edit \verb|SeeTeX/Makefile.raw|; there are several declerations you
shoudl change to reflect your local environment. The only unusual one
is {\tt FONTDESC}; see \S\ref{section:fontdesc} if you don't understand
this entry.

After than, simply say {\tt make} and you should build {\texsun} and
{\dviselect}. These programs do not depend on other files and can be
installed anywhere you like.

\end{document}

