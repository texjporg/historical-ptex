%
\subsection{Installation}
Installation of the \iptex\ and \texx\ is usually done by a system
administrator.
This section describes the installation procedure, and assumes that you
have already ``unpacked'' the files.
%
\subsubsection{Installtion}
There are three variables in the main makefile which will need to
be changed for each system.

The first, {\tt FONTDESC} is the name of a file which the \dvi\ library
uses to locate \TeX\ fonts.
This file is usually stored in {\tt /usr/local/lib/tex82/fontdesc}, but
your local conventions may differ.

The second, {\tt BINDIR} is the directory for the final binaries.
Similarly, {\tt MANDIR} is the directory for the the man pages for
the utilities.

When you have changed these variables,
execute ``{\tt make install}'' to install
\iptex, {\tt imagen1}, {\tt dviselect} and \texx.
If you modify the library routines and need to recompile them, you
should do so from the main makefile (using ``{\tt make thelib}'')
to insure that the {\tt FONTDESC} variable is passed to the
library makefile.

The font description file provides a sequence of directories to
search for fonts.
Normally, all fonts are stored in a single directory.
That directory contains the {\tt .tfm} files for the fonts
and a subdirectory for each font containing all the magsteps for that font.
The fontdesc file also determines which font storage format is used,
i.e. wither {\tt pxl}, {\tt gf} or {\tt pk}.
A sample fontdesc is shown in table \ref{table:fontdesc}.
%
\begin{table}[hbtn]
\centering
\input{fontdesc}
\caption{A sample fontdesc}
\label{table:fontdesc}
\end{table}
%
\subsection{Porting \& Customization}
\iptex\ and the \dvi\ library have been ported to the AT\&T 3B,
the DEC VAX, Sun 3, and IBM PC/RT series of computers.
\texx\ is known to work on the IBM PC/RT and Sun workstations.

When porting to a new system, the file {\tt h/types.h} may need
to be changed.
That file contains macros to properly sign-extend byte, short and long
words integers for the {\tt pk} format fonts.

Font shrinking for \texx\ is done in byte-order independent manner,
but it depends on a type (usually {\tt short}) being 16 bits long.
If you have troubles installing \texx, look at the file
{\tt previewers/dvistuff.c} to see if you need to change anything.

You may also want to change {\tt previewers/dvistuff.h} to change
the default values for margins and the device resolution of your
font family.

\iptex\ currently understands a set of \dvi\ {\tt special} commands
which allow it to draw lines using Imagen primitives.
At the Univ. of Illinois, we have a program called {\tt texpic},
a version of PIC for \TeX, which uses these specials to draw pictures.
A subset of these specials are implemented by \texx.
In the future, it is hoped that PostScript specials will be provided,
but don't hold your breath.

