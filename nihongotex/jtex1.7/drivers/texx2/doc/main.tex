\documentstyle[12pt,fullpage]{article}
%
\newcommand{\texx}{{\tt texx}}
\newcommand{\texsun}{{\tt texsun}}
\newcommand{\dvi}{{\tt dvi}}
\newcommand{\iptex}{{\tt iptex}}
%
\title{DVI Output Guide}
\author{
Dirk Grunwald \\
Dept. of Computer Science\\
University of Illinois
}
\date{Sept. 1987}
%
\begin{document}
\maketitle
\begin{abstract}
This document describes the features, use and installation of
\iptex, \texx\  and \texsun.
\iptex\ prints \dvi\ \TeX\ documents on an Imagen printer.
\texx\ is a document previewer which uses the X-11 window system.
\texsun\ is a document previewer under the SunView window system.
\iptex\ was written by Chris Torek at the University of Maryland.

Additionally, this version of \texx\ would not exist without the work of
Steve Eichin at MIT and Chris Torek at the U. of Maryland.

\texsun\ is the original work of the author.

{\bf: NOTE - this documentation is somewhat out of date, and carries no
information about TeXx2}
\end{abstract}
%
\section{Introduction}
This document is structured into two major sections.
The first of these is a users manuals for \iptex, \texx and \texsun,
and is the only section of interest for most people.
The second section covers the installation and customization of
the package.

\section{Users Guide}
In the manuals which follow, it is assumed that you have followed your
local guide to produce a \dvi\ file from your \TeX\ or \LaTeX\ document.
The first manual describes \iptex, used to print your \dvi\ file
on an Imagen printer.
The second manual describe \texx\ and \texsun.
\subsection{Iptex}
\iptex was written by Chris Torek at the University of Maryland.
The command line for invoking \iptex\ is as follows:\\
{\tt
\begin{verbatim}
iptex [ -c ] [ -d drift ] [-l] [-m mag] [-p] [-r res] [-s]
      [-X x-offset] [-Y y-offset] filename
\end{verbatim}
}

Typically, \iptex\ will be configured for your local installtion,
and you will only type {\tt verb+iptex file+} to print {\tt file.dvi}
on your Imagen.
Options other than the ones listed are passed to the local print
command, typically called {\tt ipr} or {\tt lpr}.
The default options can be over ridden by the parameters
in table \ref{table:iptex}.

\begin{table}[hntb]
\centering
\begin{tabular}{|lp{5in}|}
\hline
Option & Description \\
\hline \hline
-c &
\iptex\ is actually a shell script which executes {\tt imagen1}, which
converts your \dvi\ file, and sends the output to the printer.
Normally, \iptex\ copies the output of {\tt imagen1} to a file before
sending it to the printer.
If there is an error while translating the file, the output is not sent
to the printer, saving paper if there is a problem with your file.
The -c option forces output to be sent directly to the printer, should
that prove useful. \\

-d drift &
This rarely used parameter is used to control the {\em drift} of a document.
When your document is being converted for the Imagen, attempts are made to
correct for rounding errors which occur.
This parameter controls the maximum allowable drift of the final document.
It is never used in normal practice. You are advised to read the source
of \iptex\ if you think you need to use this. \\

-l &
This causes your document to be printed in {\em landscape mode},
that is, your normal 8.5 by 11 document is rotated to become an 11 by 8.5
document. \\

-m mag &
This applies a global magnification to the output.
You can also use this to {\em reduce} your document.
The default is {\tt -m 1000}. \\

-s &
This disables the sequence numbers which are normally printed when
your document is being processed. \\

-p &
This disables page-reversal.
You can use {\tt +p} to force page-reversal.
The default for this is {\tt -p}, which causes your document to
be printed in collated order on most Imagen printers. \\

-X offset & \\
-Y offset &
These are added to the current border width, i.e. they are
relative displacements.
Normally, your document is offset from the border by one inch.
Offsets are in thousandths of an inch and may be negative. \\
\hline
\end{tabular}
%
\caption{Options for \iptex}
\label{table:iptex}
\end{table}

\subsection{The Previewers}
\texx\ is a {\em previewer} for \dvi\ files which uses the X-11 window system.
\texsun\ is a previewer for \dvi\ files which uses the SunView window system.
Both previewers allow you to
view documents on your workstation before printing them.

Everything in the following
description applies to both \texx\ and \texsun\ unless
otherwise stated.

\subsubsection{Painting}
When you start \texx\ or \texsun , a window will be created on your display
\footnote{
	The interaction for this varies with the window manager you're using
	and between \texx\ and \texsun.
}
and your document will be {\em painted} in that window.
The size of the window corresponds to the size of the tallest and widest
page of your document.
If your display is wide enough, the window will have two {\em leaves},
otherwise it will have a single {\em leaf}.
Each leaf will be painted with a page from your document.

The previewers take some 
some pains to provide the most viewable document possible.
Normally, a \TeX\ document has an inch of margin added at the left and
top margin.
If reduction of this white space allows your document to be viewed
at a more readable size, that margin is reduced.
If you are using \texx\ or \texsun\ to preview documents you are working
on, you may find it advantageous to specify a smaller document size
to \TeX\ or \LaTeX, allowing you to preview your document at a more
readable size.

\subsubsection{Shrinks}
Because most workstations do not have a screen resolution approaching that
of a laser printer,
sacrifices must be made to provide suitable fonts.

One solution would be to generate the \TeX\ fonts at your workstation
display resolution.
This requires at least two sets of fonts, one for your printer and one
for each workstation display resolution.
Another approach is to use the high-resolution fonts for the printer
and {\em shrink} them.
This is the method used by \texx\ and \texsun,
because it reduces the number of fonts
and works with a great number of display resolutions
while still allowing the option of using workstation-specific fonts.

Fonts are shrunk using a {\em shrink factor},
which is an integer in the range of 1 to 9.
As an example, with a shrink factor of 4, a 4 by 4 set of pixels
is reduced to a single pixel on your display.
The pixel is turned on if $1/\mbox{\em blackness}$ pixels in the
original sample were turned on.
Normally, {\em blackness} is $3$, which was chosen by trial \& error,
but it can be set using the {\tt -bl} option or by setting the
{\tt Blackness} option in your defaults file.

A shrink factor of 1 is a direct view of the fonts at the font resolution.

There are usually two goals to previewing documents.
One is to view the {\em overall structure} of the document, e.g.
to check if tables, equations and figures appear at the right location.
Another is to check {\em detail structure}, e.g. to check
if the spacing around an equation looks right.
The interface for the previewers provides a {\em normal shrink} and a
{\em large shrink} view of your document.
At the normal shrink size, an entire page is visible, providing an
overall view.
At the large shrink size, details are more visible, although the entire
document is not.

\subsubsection{Differences Between \texx\ and \texsun}
When documents are being painted, the cursor icon for \texx\ changes to a
clock, indicating that you need to wait for the painting to finish.
For \texsun, the cursor changes to an hourglass.
When \texx\ is waiting for input, the cursor changes to a spraycan.
When \texsun\ is waiting for input, the cursor changes to a bulls-eye.

Unless the \texx\ window is {\em obscured} by another window,
the image is painted directly to the display.
If the \texx\ window is {\em obscured} by another window,
the image is painted to a {\em pixmap}, which is displayed when the
page is painted.
{\em You should only obscure the \texx\ window when the cursor is
a spraycan, not when it is a clock.}
If the window is covered or obscured while it is being painted, the
final image will not be correct.
However, if you obscure the window (e.g. overlay it with a
window for editing your document) while the spray scan is present,
the window will be correctly painted when you turn pages.
\texx\ does not take advantage of resized windows.

\texsun\ allows somewhat more flexibility.
Pages are painted directly to memory, and displayed when they are painted.
You can obscure the \texsun\ window at any time.
You can also enlargen the window.
While this will not change the amount of visible text for the
normal shrink size, the entire window is used to display the
detail view.

\subsubsection{Moving Around}
In both \texx\ and \texsun,
you move forward in your document by hitting
the {\tt RETURN} or {\tt NEWLINE} key.
If there is a single leaf, it will be erased and painted with the next page.
If there are two leaves, the right hand page will be put on the left leaf
and the left leaf will be painted with the next page.
In other words, hitting return turns the page.
To move backward in your document, hit the {\tt DELETE} or {\tt BACKSPACE} key.
If you type a number before hitting a key, you move forward or backward
that number of pages.

To move to a given page, type the number of the page and the ``{\tt g}''.
To quit \texx, type ``{\tt q}''.

To view a part of your document at the large shrink size,
point the cursor at the part you are interested in and hit any mouse button.
The spray can will change to a clock while the page is painted
at the larger size.
Since the entire page will not fit on the display at the larger size,
only the part you pointed to is displayed.
If your window has a single leaf, that leaf is overwritten with the
enlarged image.
In \texx,
if your window has two leaves, the enlarged image is painted
on the leaf which was not pointed to.
This allows you to see the overall structure and the detail at the same
time.
In \texsun, the entire window is always used.

If you move the mouse as you hold down the button,
the enlarged page will move past the view window, allowing you
to pan over your document.
In \texsun, the display is not updated until the mouse is inactive
for a period called the ``settle time''.
This time can be specified in seconds using the {\tt -settle} flag;
the default value is an eighth of a second.
It is also possible to set the ``significant distance'' for mouse movements
to avoid a large number of screen updates caused by ``flutter fingers''.
If the mouse is moved from the last resting point by
less than the significant distance, the display is not redrawn.
The significant distance can be set using the {\tt -signif} flag.
The argument is the percentage of movement in the large document
which should be considered significant for a redisplay.
The default value is one percent.

When you release the mouse button, the page which was previously displayed
on the overwritten leaf is restored.
If you point to another spot {\em on the same page}, the enlarged page will not
need to be recomputed.
The previewers remember the enlarged page as long as the normal sized page
is visible, i.e. if you turn forward several pages, and then back up,
you will need to recompute the enlarged page if you want to see the
detail view.

\subsubsection{Options}
The previewers
allow you to control the layout of the window, using either
the command line your window-system specfic defaults file.
Table \ref{table:previewer-options} lists the available options.
Options in {\bf bold-face} apply to both \texx\ and \texsun.
Options in {\sl slant-face} apply only to \texx,
and optioned in {\tt type} apply only to \texsun.
Normally, you invoke \texx\ by simply saying {\tt \texx\ or \texsun  yourfile}.
You invoke \texsun\ by simply saying {\tt \texsun\ yourfile}.
If ``{\tt yourfile}'' can not be found, then the suffix ``{\tt .dvi}''
is appended.
The previewer make a copy of your file, allowing you to reformat your
document while viewing other portions.
%
\begin{table}[hbt]
\centering
    \begin{tabular}{|l|c|c|p{2in}|}
	\hline
	XDefault	& Command Line & Argument Type & Meaning \\
	\hline \hline

	{\bf ReverseVideo}	& -rv
	& {\em none } & Display document in reverse video \\

	{\sl BorderWidth}	& -bw
	&  {\em integer } & Change the width of the display border. \\

	{\sl ForeGround}	& -fg
	& {\em color}	& Change the foreground display color\\ 

	{\sl BackGround}	& -bg
	& {\em color}	& Change the background display color\\ 

	{\sl HighLight}	& -hl
	& {\em color}	& Change the highlight color \\

	{\sl Border}		& -bd
	& {\em color}	& Change the border color \\

	{\sl Mouse}		& -ms
	& {\em color}	& Change the mouse color \\

	{\bf NormalShrink}	& -ns
	& {\em integer} & Change the shrink size for the overview display \\

	{\bf LargeShrink}	& -ls
	& {\em integer} & Change the shrink size for the detail display \\

	{\bf Blackness}		& -bl
	& {\em integer}	& Change the blackness threshold for shrunken fonts \\

	{\bf Leaves}		& -l
	& {\em integer} & Change the number of leaves uses (1 or 2) \\

	{\bf TopMargin}	& -tm
	& {\em float}	& Set the top margin in inches \\

	{\bf SideMargin}	& -sm
	& {\em float}	& Set the left side margin in inches \\

	{\tt SettleTime}	& -settle
	& {\em float}	& Set the mouse tracking settle time in seconds \\

	{\tt SignificantDistance}	& -signif
	& {\em float}	& Set the significant distance for mouse tracking
			  as a percentage of the large document size \\
	{\tt AutoReload}		& -autoReload
	& {\em none}	& Automatically reload the file when the window is
			  exposed or opened \\

	{\bf Dpi}		& -dpi
	& {\em integer} & Set the desired font resolution in dots per inch \\
	\hline
    \end{tabular}
\caption{Options For \texx\ and \texsun}
\label{table:previewer-options}
\end{table}

\subsubsection{Remembering Where You Were}
Currently, this section only applies to \texsun.

Since the normal use of \texsun\ involves the repeated viewing of a document
to fix formatting problems,
it would be nice to be able to return to the page you were viewing when
you left \texsun.
One way to do this is to {\em re-load} the document you're viewing.
However, if you freak out after too much editing \& run off for some
crumpets or something, it'd be nice to have \texsun\ remember where you
left off.

If you define the environment variable ``{\tt TEXSUN\_STATE}'' as the name
of a file, a {\em viewing state} will be written to that file when you
exit \texsun. 
The next time you start \texsun, this state will be added to your list of
command-line arguments before any other argument.
This allows you to override the saved state.
If the first argument is {\tt -new}, the saved state is ignored.

If you choose an absolute path name for {\tt TEXSUN\_STATE},
you can invoke \texsun\ from a menu, but all your previewer
sessions will use the same state file.
Choosing a relative path name (e.g. ``{\tt ./.texsun-state}'') allows
you to have a state file for each different directory, but it's
difficult to convince the SunView environment to start a
tool in a given directory.

\subsubsection{Reloading your document}
Since the normal editing cycle, crumpets not withstanding, is to edit, format
view and edit again, you can {\em reload} your document without restarting
the previewer.
One way to do this is to hit 'R' while viewing the document.
If you're using a Sun, you can use the somewhat more civilized 'AGAIN' key.
Someday, all of this will use menus.

\subsubsection{Odds \& Ends}
\texsun\ has had a bit more development put into it, mainly because it gets
more use. \texsun\ undertands the 'OPEN' and 'EXPOSE' buttons.
The 'AGAIN' button reloads your document.
The 'DELETE' button removes the window.
The \texsun\ banner displays the {\em document size} in inches. 
This is the size of the document you see on your display {\em if it were
actually printed}.

If you have free time, here's a few things you could add to either of \texsun\ 
or \texx.
First, it'd be nice if \texx\ and \texsun had an equivilent user interface.
This simply involves a little hackery to \texx.
Other useful widgets would be
\begin{itemize}
\item	An optional ruler to the right of the first page and across the bottom
	of each page. This would make alignments \& tweaking so much easier
\item	Menus instead of key presses.
\item	A (menu selected) list of information about your document.
\item	Support for shaded areas in \texx.
\item	A font shrinking method which does a bit more anti-aliasing.
\end{itemize}

If you make improvements, please tell the world \& send the improved code to
user 'grunwald' at host 'm.cs.uiuc.edu'.


\newpage
\section{Installation Guide}
This section of the guide should be ignored by users.
It pertains to installing \iptex\ and \texx\,\texsun, and porting
the code to new hardware.
%
\subsection{Installation}
Installation of the \iptex\ and \texx\ is usually done by a system
administrator.
This section describes the installation procedure, and assumes that you
have already ``unpacked'' the files.
%
\subsubsection{Installtion}
There are three variables in the main makefile which will need to
be changed for each system.

The first, {\tt FONTDESC} is the name of a file which the \dvi\ library
uses to locate \TeX\ fonts.
This file is usually stored in {\tt /usr/local/lib/tex82/fontdesc}, but
your local conventions may differ.

The second, {\tt BINDIR} is the directory for the final binaries.
Similarly, {\tt MANDIR} is the directory for the the man pages for
the utilities.

When you have changed these variables,
execute ``{\tt make install}'' to install
\iptex, {\tt imagen1}, {\tt dviselect} and \texx.
If you modify the library routines and need to recompile them, you
should do so from the main makefile (using ``{\tt make thelib}'')
to insure that the {\tt FONTDESC} variable is passed to the
library makefile.

The font description file provides a sequence of directories to
search for fonts.
Normally, all fonts are stored in a single directory.
That directory contains the {\tt .tfm} files for the fonts
and a subdirectory for each font containing all the magsteps for that font.
The fontdesc file also determines which font storage format is used,
i.e. wither {\tt pxl}, {\tt gf} or {\tt pk}.
A sample fontdesc is shown in table \ref{table:fontdesc}.
%
\begin{table}[hbtn]
\centering
\input{fontdesc}
\caption{A sample fontdesc}
\label{table:fontdesc}
\end{table}
%
\subsection{Porting \& Customization}
\iptex\ and the \dvi\ library have been ported to the AT\&T 3B,
the DEC VAX, Sun 3, and IBM PC/RT series of computers.
\texx\ is known to work on the IBM PC/RT and Sun workstations.

When porting to a new system, the file {\tt h/types.h} may need
to be changed.
That file contains macros to properly sign-extend byte, short and long
words integers for the {\tt pk} format fonts.

Font shrinking for \texx\ is done in byte-order independent manner,
but it depends on a type (usually {\tt short}) being 16 bits long.
If you have troubles installing \texx, look at the file
{\tt previewers/dvistuff.c} to see if you need to change anything.

You may also want to change {\tt previewers/dvistuff.h} to change
the default values for margins and the device resolution of your
font family.

\iptex\ currently understands a set of \dvi\ {\tt special} commands
which allow it to draw lines using Imagen primitives.
At the Univ. of Illinois, we have a program called {\tt texpic},
a version of PIC for \TeX, which uses these specials to draw pictures.
A subset of these specials are implemented by \texx.
In the future, it is hoped that PostScript specials will be provided,
but don't hold your breath.


\end{document}
