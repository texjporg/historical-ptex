\subsection{Iptex}
\iptex was written by Chris Torek at the University of Maryland.
The command line for invoking \iptex\ is as follows:\\
{\tt
\begin{verbatim}
iptex [ -c ] [ -d drift ] [-l] [-m mag] [-p] [-r res] [-s]
      [-X x-offset] [-Y y-offset] filename
\end{verbatim}
}

Typically, \iptex\ will be configured for your local installtion,
and you will only type {\tt verb+iptex file+} to print {\tt file.dvi}
on your Imagen.
Options other than the ones listed are passed to the local print
command, typically called {\tt ipr} or {\tt lpr}.
The default options can be over ridden by the parameters
in table \ref{table:iptex}.

\begin{table}[hntb]
\centering
\begin{tabular}{|lp{5in}|}
\hline
Option & Description \\
\hline \hline
-c &
\iptex\ is actually a shell script which executes {\tt imagen1}, which
converts your \dvi\ file, and sends the output to the printer.
Normally, \iptex\ copies the output of {\tt imagen1} to a file before
sending it to the printer.
If there is an error while translating the file, the output is not sent
to the printer, saving paper if there is a problem with your file.
The -c option forces output to be sent directly to the printer, should
that prove useful. \\

-d drift &
This rarely used parameter is used to control the {\em drift} of a document.
When your document is being converted for the Imagen, attempts are made to
correct for rounding errors which occur.
This parameter controls the maximum allowable drift of the final document.
It is never used in normal practice. You are advised to read the source
of \iptex\ if you think you need to use this. \\

-l &
This causes your document to be printed in {\em landscape mode},
that is, your normal 8.5 by 11 document is rotated to become an 11 by 8.5
document. \\

-m mag &
This applies a global magnification to the output.
You can also use this to {\em reduce} your document.
The default is {\tt -m 1000}. \\

-s &
This disables the sequence numbers which are normally printed when
your document is being processed. \\

-p &
This disables page-reversal.
You can use {\tt +p} to force page-reversal.
The default for this is {\tt -p}, which causes your document to
be printed in collated order on most Imagen printers. \\

-X offset & \\
-Y offset &
These are added to the current border width, i.e. they are
relative displacements.
Normally, your document is offset from the border by one inch.
Offsets are in thousandths of an inch and may be negative. \\
\hline
\end{tabular}
%
\caption{Options for \iptex}
\label{table:iptex}
\end{table}
