\documentstyle[postscript,fullpage,leftpar]{article}
\sloppy

\begin{document}

\title{Including Figures into Typeset Documents}

\author{Neil Hunt \\
	Teleos Research \\
	576 Middlefield Road \\
	Palo Alto \\
	California 94301}
\maketitle

\section{Introduction}

While typesetting languages like \TeX\ \cite{Knuth:texbook}
and \LaTeX\ \cite{Lamport:latex} have
facilities for making line drawings by describing them in
words,
the procedure is very slow and error prone,
and the quality of the results can be disappointing.
Such illustrations are more quickly prepared using
a graphical editor.

An alternative is to prepare the whole document
using a
``What You See Is What You Get''
system such as Macdraw or FrameMaker.
Unfortunately these have their own limitations --
typesetting mathematics in these systems is very hard,
for example.

A lot of users prefer to prepare the {\em text} of documents
using a typesetting language,
and the {\em figures} using a graphical editor.
The cut and paste operation which is then required to put the
final document together is very undesirable.

In all three of these cases, the problems with incorporating
figures into a document make authors
reluctant to add figures when they would be helpful,
and delay making corrections and changes.

I describe a system
by which separate figures
can be incorporated into a typeset document
at the time of printing.
The system permits a variety of ways of sizing the illustrations,
either scaling the illustration to the
space available in the text,
or setting the size of the space in the text
to match the original size of the figure.

Apart from illustrations created on a graphical editor,
digitised photographs, processed images, screendumps, and a variety
of other types of figures can be merged into
a document by this system.

The main issues to be addressed by such a system are:
\begin{itemize}
\item	The size of the figure and the space left in the flow of
	text must be the same.
\item	When the figure is updated, its size may change;
	the size of the space left in the text must change to match.
\item	If the author is responsible for measuring the size of the
	illustration and typing it into the typesetter source text,
	most of the figures will {\em not} fit in the spaces left for them.
\item	When a user creates a figure in a graphical editor,
	he draws it with a particular size.
	Usually, this will be the size it should have in the final document.
	Sometimes it is appropriate to scale figures,
	either by a fixed amount, or perhaps
	to make all the figures as wide as the width of the column of text.
	A useful system {\em must} provide for the size of the illustration
	to be set in a variety of ways.
\end{itemize}

The problem breaks down into two parts:
First, there is the problem of merging the figure description
and the textual description into a single stream sent to the printer.
Second, there is the problem of communicating the size and scaling
information between the systems used to prepare the illustrations,
and the typesetter.

Fortunately, much of the problem is solved automatically
by the use of PostScript \cite{Adobe:postscript} as a page description language
for both the text and the illustrations.
It is relatively easy to merge the PostScript streams,
and the PostScript itself provides the medium for
communicating the document sizes,
by means of the meta comment
\verb|%%BoundingBox|.
We shall now examine how to put the various pieces together
to make a useful system.

\section{Merging PostScript Figures into \TeX\ and \LaTeX}

The \TeX\ typesetter
generates device independent {\tt DVI} output,
which is subsequently converted into a printer language.
By focusing on PostScript as the printer language,
we lose the device independence;
however, with the proliferation of PostScript printers,
this is becoming less of an issue.

The \verb|\special| command of \TeX\
allows an arbitrary string to be passed from the
document source language into the {\tt DVI} file,
where it can be interpreted as a command to merge in a separate file.
The focus of the merging operation is thus the {\tt dvips} postprocessor
which performs the conversion from {\tt DVI} to PostScript.
I defined a \verb|\special| command
which conveys a filename, a width, and a height to the {\tt dvips}
program.
{\tt DVI} has the notion of a {\em current position}
on the page; this supplies the position at which the figure is to
be placed, which the width and height from the
\verb|\special|
command used to control the size of the figure.

For the figures to be included,
I chose to use a standard called Encapsulated PostScript (EPS),
which is a set of guidelines on how to structure a PostScript
program to facilitate this kind of merging.
The most important feature of EPS in this application
is the meta comment
``\verb|%%BoundingBox:| $x1$ $y1$ $x2$ $y2$''
which delimits the rectangle on the page within which all marks
made by the program will lie,
expressed in the coordinate system in effect when the program
is executed.
By inserting appropriate {\tt translate} and {\tt scale} commands
into the PostScript ahead of the program,
the figure may be fitted into any part of the page.
The other requirement for merging is to cancel the
command {\tt showpage} of the merged document,
which would otherwise cause the page to be printed
prematurely,
before all the required marks had been made upon it.

Sadly, the bounding box meta comment is not required for a conforming
EPS program, and many programs do not produce one.
Without some way of knowing the region of the page covered by
the figure, it is impossible to meet the goals stated above
for including figures in documents.
For figures produced by FrameMaker,
which do not include a bounding box comment,
the size can be deduced by scanning the PostScript
for a command sequence defining a ``frame'' within which the
figure can be drawn.
If the user adopts the convention of always drawing within a frame,
this works satisfactorily.
Another sad example is MacDraw,
which produces PostScript which is far from conforming to EPS conventions,
and so far all attempts to include Macdraw diagrams have failed.
Fortunately Adobe Illustrate is a better program which generates
conforming EPS, with a bounding box comment.

I added code to {\tt dvips} which responds to a
\verb|\special|
command.
The indicated PostScript file is opened,
the
\verb|%%BoundingBox|
comment is located and read,
the translation and scaling required to cause
the corners of the bounding box to match the position and size
is determined.
PostScript is then emitted which redefines the {\tt showpage}
command as null and performs the transformations,
and the included PostScript program is copied to the output.

I then created a macro\\
\verb|\postscriptbox{width}{height}{filename}|
which creates a \TeX\ ``box'' of the indicated width and height,
with the current point set to the bottom left corner.
A box is a primitive in \TeX\ which may be used exactly
like a character; it can be included in a line of text,
stacked in an equation or array,
or included within a
\verb|\center| and a \verb|\figure|
environment to make a centred, floating figure.

This solves the first part of the problem: merging figures
into the \TeX\ or \LaTeX\ document.
It also ensures that the figure is always scaled
to the size of the space left in the flow of text in the document
(the size of the box).
However, the communication of size which it affords is one directional;
there is no way for the user to include a figure
at its natural or designed size.
Of course, this missing capability is the one most required
for including figures.

\def\mug{\raisebox{-0.5ex}{\postscriptbox{2ex}{2.5ex}{pstex-figs/coffee}}}

The \verb|\postscriptbox| macro on its own is useful for defining new
symbols or characters without going to {\sc METAFONT}
\cite{Knuth:tex-metafont}.
For example, the macro
\verb|\def\mug{\postscriptbox{2ex}{2.5ex}{coffee}}|
defines a coffee mug character (\,\mug) which tracks the size of
the current font:
{\tiny tiny mug \mug}
{\small small mug \mug}
{\normalsize normal mug \mug}
{\Large large mug \mug}.

\section{Communicating Size from Figure to \TeX\ or \LaTeX}

\begin{figure*}
  \begin{center}
    \psbox[width=0.95\textwidth]{pstex-figs/flow}
  \end{center}
  \caption{The process of typesetting a \LaTeX\ document with PostScript
    figures.
    This figure was included at 95\% of the width of the text.}
  \label{fig:flow}
\end{figure*}
The second part of the problem is communicating
the original design size of the figure from the system
on which it was created into \TeX.
Clearly this information is carried in the PostScript
as the bounding box;
however, there is no way to get \TeX\ or \LaTeX\ to parse
the PostScript to obtain this data
without making significant extensions
to the typesetter program itself,
which is not practical.

The use of a preprocessor is the obvious solution.
Such a tool could parse the typesetter source text
to detect a meta-command indicating the inclusion of a figure,
read the figure file to obtain its size,
and replace the meta command in the source text stream
with a
\verb|\postscriptbox| command.

There are a few problems with this approach.
First, it is not easy to convince \TeX\ or \LaTeX\ to
operate in a pipeline appropriate for a preprocessor.
If this could be solved, the preprocessor would
have to understand not only its own meta command,
but also recognise and act upon
\verb|\input|, \verb|\include|, and \verb|\includeonly|
commands.
It would also have to understand conditional commands
and indeed the full syntax of defining new macros,
in case some other command of interest is buried within another macro.

Fortunately, an alternative mechanism exists,
as exploited by other facilities such as
BiB\TeX.
The procedure is to use the {\tt .aux} file to drive the
processor.
This file is created by \LaTeX\ as it processes the document;
there is no hierarchy (commands are not buried within other macros)
and all conditional expressions are resolved.
The job of the processor is now much simpler,
as it only has to recognise the figure inclusion meta command,
and the the \verb|\input| command.
(\TeX\ does not automatically generate such a file;
however it is easy to add the macros required to make
it behave in a similar way as \LaTeX,
and enable the same figure facilities to be used.)

I have created a program called {\tt pstex}
which scans the {\.aux} file created by \LaTeX,
locating figure inclusion meta commands.
When it finds a figure to be included,
it reads the PostScript of the figure
to determine its size,
and writes an additional file
called the {\tt .tps} file,
which contains just a \verb|\postscriptbox| command,
with the size filled in according to the size read from the PostScript.

To go with this, there is an additional \LaTeX\ macro,
called \verb|\psbox[options]{filename}|
This macro performs two tasks;
first it causes an entry to be made in the
{\tt .aux} file containing the options string and the filename,
and then it searches for the {\tt .tps} file with a name corresponding
to that in the argument.
On the first run of \LaTeX, there will be no such file,
and the \verb|\psbox| macro will generate an \verb|\fbox|
(a framed box) as a placeholder.
After {\tt pstex} has been run over the {\tt .aux} file,
in subsequent passes of \LaTeX\ the \verb|\psbox|
macro {\em will} find the {\tt .tps} file,
containing the primitive \verb|\postscriptbox|
command with the sizes as derived from the PostScript,
which it then executes.

This mechanism allows an author to include a figure file
at the size at which it was designed,
without having to type that size into the source text
for the typesetter at any time.
Furthermore,
whenever the figure is altered,
any size change is reflected in the document simply
by making one pass with {\tt pstex} and one pass with \LaTeX.

\subsection{The Options String}

\begin{figure*}
  \begin{center}
    \psbox[scale=0.5]{pstex-figs/screen}
  \end{center}
  \caption{Example screendump, included at 0.5 $\times$ nominal scale.}
  \label{fig:screen}
\end{figure*}

The \verb|\psbox| command accepts an optional argument,
which is passed on directly to {\tt pstex}.
This argument is used to control how {\tt pstex}
converts the size obtained from the PostScript
into the width and height in the \verb|\postscriptbox| command.
Currently {\tt pstex} understands options
{\tt width=X},
{\tt height=X},
{\tt scale=Y},
and {\tt aspect=Y},
where {\tt X} is a string or macro which expands into any length
command valid
within a \verb|\setlength| command
(such as \verb|2in| or \verb|0.9\columnwidth|),
and {\tt Y} is a floating point number or a macro which expands into
a floating pointer number.
The {\tt width} and {\tt height} options override the
natural width and height obtained from the PostScript file;
if only one is specified,
then the natural aspect ratio of the PostScript is preserved,
unless overridden by an {\tt aspect} option.
The {\tt scale} option scales both dimensions whether obtained
from the PostScript, or overridden by one of the other options.

Since the options string is passed unchanged to {\tt pstex},
new capabilities can be added very easily;
for example an option could be made which adds a fixed amount of
additional margin around each figure.
A more elaborate option could cause {\tt pstex}
to include a conditional in flagged {\tt .tps}
files so that large time consuming figures
could be omitted from the printing process on draft printings.

An advantage of having the width and height be arbitrary
\TeX\ length specifications is that they may contain
{\em glue}, or stretchable dimensions,
allowing \TeX\ to vary the lengths to fit optimally
into the page layout.
An additional type of option which might be implemented in {\tt pstex}
could allow a degree of stretch to be added to the natural
size of the figure.

\subsection{Examples}

Figure \ref{fig:flow} shows the complete process
of typesetting a document starting from
\LaTeX\ source text and PostScript source figures,
producing a final printed output.
Heavy lines show the major iteration which is repeated
as the document is modified and refined.
This figure was created using FrameMaker,
and included with the command
\verb|\psbox[width=0.95\textwidth]{pstex-figs/flow}|.

Figure \ref{fig:screen} is a portion of a screendump from a Sun
showing the same process,
included with the command \verb|\psbox[scale=3]{pstex-figs/screen}|,
making each pixel of the display screen appear
as a block of $3 \times 3$ pixels on
the final output.

\begin{table*}[p]
  \begin{center}
    \begin{tabular}{lp{3.5in}}
      \hline
      \\
      \multicolumn{2}{c}{\Large\bf Quick Reference Chart} \\
      \\
      \hline
      \\
      \verb|\postscriptbox{1.2in}{2in}{file}| &
	 Incorporates figure {\tt file.ps} into the document as a box
	(character-like object) with size 1.2 inches $\times$ 2 inches. \\
      \\
      \verb|\psbox{file}| &
	Incorporates figure {file.ps} into the document as a box
	with its natural or design size. \\
      \\
      \verb|\psbox[scale=0.7]{file}| &
	Incorporates figure {file.ps} into the document as a box
	with its natural or design size scaled by 0.7. \\
      \\
      \verb|\psbox[scale=2]{file}| &
	Incorporates figure {file.ps} into the document as a box
	with its natural or design size scaled by 2;
	integer scale factors are useful for bitmaps which don't scale
	well with factional scales. \\
      \\
      \verb|\psbox[width=3cm]{file}| &
	Incorporates figure {file.ps} into the document as a box
	with width 3cm, and height in accordance
	with its aspect ratio. \\
      \\
      \verb|\psbox[width=\columnwidth]{file}| &
	Incorporates figure {file.ps} into the document as a box
	stretched as wide as the column,
	and with its natural aspect ratio;
	useful for single column figures. \\
      \\
      \verb|\psbox[width=\textwidth]{file}| &
	Incorporates figure {file.ps} into the document as a box
	stretched as wide as the page,
	and with its natural aspect ratio;
	useful for double column figures. \\
      \\
      \verb|\psbox[width=\textwidth,scale=.9]{file}| &
	Incorporates figure {file.ps} into the document as a box
	stretched as wide as the page, and then scaled by 0.9. \\
      \\
      \verb|\psbox[height=\textheight]{file}| &
	Incorporates figure {file.ps} into the document as a box
	stretched as high as the page, with its natural aspect
	ratio.  Useful for tall thin figures. \\
      \\
      \verb|\psbox[width=\textwidth,aspect=1.0]{file}| &
	Incorporates figure {file.ps} into the document as a box
	stretched as wide as the page, with unity aspect ratio. \\
      \\
      \hline
      \\
	\verb|\begin{figure}| \\
        \verb|  \begin{center}| \\
        \verb|    \psbox{file}| \\
        \verb|    \caption{Title of figure}|
	  & Macros for a floating centered figure. \\
        \verb|    \label{fig:file}| \\
        \verb|  \end{center}| \\
        \verb|\end{figure}| \\
      \\
      \hline
    \end{tabular}
  \end{center}
\end{table*}

\section{Conclusions}

I have described a system for merging PostScript figures into
documents prepared under \LaTeX\ and \TeX.
The author has a great deal of flexibility in
specifying how the figures are to be sized,
and in particular can specify that a figure be included
at its natural or design size, or some scale multiple thereof.
The figure always fits the space left in the flow of text,
without any need for the user to copy sizes from
the program used to generate the figure into the document source text,
and changes to the figure which alter its size
are reflected in the document automatically when it is next typeset.

These facilities are in use at SPAR,
and are available immediately for distribution to other Schlumberger sites.
Also available are some programs for converting digitised photographs,
screendumps, and other images into EPS style PostScript programs
with suitable bounding box comments.

\begin{thebibliography}{xx}

\bibitem[Adobe Systems Inc., 1985]{Adobe:postscript}
Adobe Systems Inc.
{\it PostScript Language Reference Manual}.
Addison Wesley, 1985.

\bibitem[Knuth, 1979]{Knuth:tex-metafont}
Donald~E.~Knuth.
{\it \TeX\ and METAFONT}.
Digital Press, 1979.

\bibitem[Knuth, 1984]{Knuth:texbook}
Donald~E.~Knuth.
{\it The \TeX\ book}.
American Mathematical Society, 1984.

\bibitem[Lamport, 1986]{Lamport:latex}
Leslie~Lamport.
{\it \LaTeX: A Document Preparation System}.
Addison Wesley, 1986.

\end{thebibliography}

\end{document}
